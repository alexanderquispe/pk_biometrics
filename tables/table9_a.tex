\begin{table}[H] \begin{center} \protect \caption{\label{tab:table9a} Extension II: Spillovers in Rank Improvement to Neighboring Municipalities}
\noindent\resizebox{\textwidth}{!}{ \begin{threeparttable}
\begin{tabular}{lcccccc}
\hline
\hline
                    &\multicolumn{1}{c}{(1)}   &\multicolumn{1}{c}{(2)}   &\multicolumn{1}{c}{(3)}   &\multicolumn{1}{c}{(4)}   &\multicolumn{1}{c}{(5)}   &\multicolumn{1}{c}{(6)}   \\
\hline
RD\_Estimate         &       1.874** &       2.256** &       2.153***&       2.242***&       2.454** &       2.454** \\
                    &     (0.861)   &     (0.940)   &     (0.753)   &     (0.841)   &     (1.059)   &     (1.059)   \\
\midrule
Bandwidth type      &         MSE   &         CER   &     MSE TWO   &     CER TWO   &         MSE   &         CER   \\
Bandwidth size      &      12.567   &       9.812   &15.931,9.137   &12.439,7.134   &       10.35   &       10.35   \\
Polynomial          &      Linear   &      Linear   &      Linear   &      Linear   &   Quadratic   &   Quadratic   \\
N left              &        5210   &        3475   &        5633   &        5122   &        5633   &        5633   \\
N rigth             &        6192   &        5346   &        4940   &        3998   &        6023   &        6023   \\
95\% Conf. Interval &[.073 ; 4.038]   &[.343 ; 4.374]   &[.695 ; 4.098]   &[.603 ; 4.234]   &[.382 ; 4.644]   &[.382 ; 4.644]   \\
Elections           &          57   &          45   &          43   &          36   &          49   &          49   \\
Mean (SD)           &0.30 (10.38)   &0.36 (10.53)   &0.39 (10.31)   &0.20 (10.64)   &0.35 (10.44)   &0.35 (10.44)   \\
\hline
\end{tabular}
\begin{tablenotes}
\begin{small}
\item[1]
This
table
reports
estimates
from
sharp
regression
discontinuity
designs
that
relate
the
gender
of
the
incumbent
mayor
in
a
given
municipality
to
a
measure
for
the
performance
of
female
candidates
in
the
election
for
the
local
council
in
neighboring
municipalities.
\item[2]
The
sample
covers
all
female
candidates
for
local
council
elections
in
Hessian
municipalities
that
neighbor
a
municipality
where
in
the
last
mayor
election
the
top
two
candidates
were
of
opposite
gender.
\item[3]
The
treatment
variable
is
a
dummy
that
is
1
if
the
margin
of
victory
of
the
female
candidate
in
a
neighboring
municipality
in
the
last
mixed-gender
mayor
election
was
positive.
\end{small}
\end{tablenotes} \end{threeparttable} } \end{center} \end{table}
