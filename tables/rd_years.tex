\begin{table}[H] \begin{center} \protect \caption{\label{tab:newtable4} New Question:Rank Improvement of Female Candidates by years}
\noindent\resizebox{\textwidth}{!}{ \begin{threeparttable}
\begin{tabular}{lccccc}
\hline
\hline
                    &\multicolumn{1}{c}{(1)}   &\multicolumn{1}{c}{(2)}   &\multicolumn{1}{c}{(3)}   &\multicolumn{1}{c}{(4)}   &\multicolumn{1}{c}{(5)}   \\
\hline
female\_mayor        &       4.141***&       4.245***&      -1.197   &       3.035** &       -1.249        \\
                    &     (0.708)   &     (1.040)   &     (1.119)   &     (1.240)   &  (1.219)              \\
\midrule
Bandwidth type      &         CCT   &         CCT   &         CCT   &         CCT   &         MSE   \\
Year                &       32.52   &       19.26   &       17.89   &       18.57   &      17.993   \\
Bandwidth size      &      Linear   &      Linear   &      Linear   &      Linear   &      Linear   \\
Polynomial          &        2001   &        2006   &        2011   &        2016   &        2011   \\
N                   &         801   &         706   &         683   &         797   &        2020   \\
Elections           &          15   &          17   &          21   &          28   &          22   \\
Municipalities      &          15   &          17   &          21   &          28   &               \\
Mean (SD)           & 1.35 (8.60)   &1.74 (11.73)   &-0.38 (11.10)   &-0.59 (12.18)   &-0.35 (11.17)   \\
\hline
\end{tabular}
\begin{tablenotes}
\begin{small}
\item[1]
This
table
reports
estimates
from
sharp
regression
discontinuity
designs
that
relate
the
gender
of
the
incumbent
mayor
to
the
initial
placement
of
female
candidates
on
the
party
lists
in
the
election
for
the
local
council
in
the
same
municipality.
\item[2]
The
dependent
variable
is
the
normalized
initial
rank
of
a
female
candidate.
\item[3]The
treatment
variable
is
a
dummy
that
is
1
if
the
margin
of
victory
of
the
female
candidate
in
the
last
mixed-gender
mayor
election
was
positive.
\end{small}
\end{tablenotes} \end{threeparttable} } \end{center} \end{table}
