\begin{table}[H] \begin{center} \protect \caption{\label{tab:table5_a} Mechanism II: Share of Women on Party Lists}
\noindent\resizebox{\textwidth}{!}{ \begin{threeparttable}
\begin{tabular}{lcccccc}
\hline
\hline
                    &\multicolumn{1}{c}{(1)}   &\multicolumn{1}{c}{(2)}   &\multicolumn{1}{c}{(3)}   &\multicolumn{1}{c}{(4)}   &\multicolumn{1}{c}{(5)}   &\multicolumn{1}{c}{(6)}   \\
\hline
RD\_Estimate         &       0.033   &       0.021   &      -0.009   &      -0.013   &       0.001   &      -0.010   \\
                    &     (0.031)   &     (0.031)   &     (0.058)   &     (0.059)   &     (0.059)   &     (0.060)   \\
\midrule
Bandwidth type      &         MSE   &         CER   &         MSE   &         CER   &         MSE   &         CER   \\
Bandwidth size      &      15.691   &      12.247   &       9.976   &       7.928   &       9.360   &       7.439   \\
Polynomial          &      Linear   &      Linear   &      Linear   &      Linear   &      Linear   &      Linear   \\
N left              &        3620   &        3248   &        1040   &         952   &        1040   &         952   \\
N rigth             &        4741   &        4071   &        1583   &        1385   &        1583   &        1341   \\
95\% Conf. Interval &[-.038 ; .101]   &[-.052 ; .091]   &[-.148 ; .107]   &[-.145 ; .11]   &[-.136 ; .12]   &[-.147 ; .116]   \\
Elections           &          65   &          55   &          46   &          39   &          45   &          37   \\
Mean (SD)           & 0.30 (0.46)   & 0.29 (0.46)   & 0.29 (0.45)   & 0.29 (0.46)   & 0.29 (0.45)   & 0.29 (0.46)   \\
\hline
\end{tabular}
\begin{tablenotes}
\begin{small}
\item[1]
This
table
reports
estimates
from
sharp
regression
discontinuity
designs
that
relate
the
gender
of
the
incumbent
mayor
to
the
gender
of
candidates
on
party
lists
in
the
election
for
the
local
council
in
the
same
municipality.
\item[2]
The
dependent
variable
is
a
dummy
that
is
1
if
a
candidate
is
female.
\item[3]
The
sample
covers
all
candidates
for
local
council
elections
in
Hessian
municipalities
where
in
the
last
mayor
election
the
top
two
candidates
were
of
opposite
gender
(mixed
gender
races).
\end{small}
\end{tablenotes} \end{threeparttable} } \end{center} \end{table}
