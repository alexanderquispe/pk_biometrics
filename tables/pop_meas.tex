\begin{table}[H] \begin{center} \protect \caption{\label{tab:newtable1} New Question: Initial List Placement of Female Candidates and proxy of 
"traditional
gender
norms"}
\noindent\resizebox{\textwidth}{!}{ \begin{threeparttable}
\begin{tabular}{lcccccc}
\hline
\hline
                    &\multicolumn{1}{c}{(1)}   &\multicolumn{1}{c}{(2)}   &\multicolumn{1}{c}{(3)}   &\multicolumn{1}{c}{(4)}   &\multicolumn{1}{c}{(5)}   \\
\hline
female mayor        &       3.723   &       5.999   &       4.204   &       3.399   &       3.898   \\
                    &     (6.182)   &     (6.917)   &     (4.036)   &     (5.842)   &     (6.845)   \\
log of Population    &      -1.342** &       0.246   &      -1.442** &      -1.400** &      -1.376***\\
                    &     (0.603)   &     (1.225)   &     (0.561)   &     (0.594)   &     (0.529)   \\
\midrule
Bandwidth type      &         CCT   &       CCT/2   &        2CCT   &          IK   &         CCT   \\
Bandwidth size      &       14.52   &        7.26   &       29.04   &       17.17   &       26.12   \\
Polynomial          &      Linear   &      Linear   &      Linear   &      Linear   &   Quadratic   \\
N                   &        2406   &        1434   &        3807   &        2574   &        3721   \\
Elections           &          63   &          36   &         113   &          69   &         110   \\
Municipalities      &          47   &          30   &          79   &          52   &          78   \\
Mean (SD)           &37.28 (25.71)   &36.98 (25.78)   &37.72 (25.88)   &37.32 (25.77)   &37.65 (25.91)   \\
\hline
\end{tabular}
\begin{tablenotes}
\begin{small}
\item[1]
This
table
reports
estimates
from
sharp
regression
discontinuity
designs
that
relate
the
gender
of
the
incumbent
mayor
to
the
initial
placement
of
female
candidates
on
the
party
lists
in
the
election
for
the
local
council
in
the
same
municipality.
\item[2]
The
dependent
variable
is
the
normalized
initial
rank
of
a
female
candidate.
\item[3]
The
sample
in
panel
A
covers
all
female
candidates
for
local
council
elections
in
Hessian
municipalities
where
in
the
last
mayor
election
the
top
two
candidates
were
of
opposite
gender
(mixed
gender
races).
\end{small}
\end{tablenotes} \end{threeparttable} } \end{center} \end{table}
